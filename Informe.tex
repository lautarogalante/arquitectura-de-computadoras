\documentclass[a4paper,twoside]{article}
\usepackage[utf8]{inputenc}
\usepackage[spanish]{babel}
\usepackage[a4paper, margin=1in]{geometry}
\usepackage{graphicx}
\usepackage{listings}
\usepackage{algorithm2e}
\usepackage{setspace}
\usepackage{microtype}
\usepackage{float}
\usepackage{color}
\usepackage{stringstrings}
\usepackage{adjustbox}

\definecolor{codegreen}{rgb}{0,0.6,0}
\definecolor{codegray}{rgb}{0.5,0.5,0.5}
\definecolor{codepurple}{rgb}{0.30,0,0.30}
\definecolor{backcolour}{rgb}{0.95,0.95,0.92}
\definecolor{codeblue}{rgb}{0,0,1}
\definecolor{codebordo}{rgb}{0.5,0,0}

\newcommand{\codebordo}[1]{\textcolor{codebordo}{#1}}
\newcommand{\codegreen}[1]{\textcolor{codegreen}{#1}}
\newcommand{\codeblue}[1]{\textcolor{codeblue}{#1}}
\newcommand{\codegray}[1]{\textcolor{codegray}{#1}}
\newcommand{\codepurple}[1]{\textcolor{codepurple}{#1}}


\lstdefinelanguage{8086}{
  morekeywords=[1]{db, mov, sub, offset, int, inc, dec, call, add, cmp, div, jmp, jne, jg, ORG, DATA, CODE, MAIN, PRINT, BINARYSEARCH, PROC, SEARCHLOOP, SEARCHEND, second, third, RET, ENDP, END},
  morekeywords=[2]{ax, ah, al, bx, bh, bl, cx, ch, cl, dx, dh, dl, si},
  morecomment=[l]{;},
  morestring=[b]',
  morestring=[b]",
  sensitive=false
}


\lstdefinestyle{mystyle}{
    backgroundcolor=\color{backcolour},   
    commentstyle=\color{codegreen},
    keywordstyle=[1]\color{codeblue},
    keywordstyle=[2]\color{codebordo},
    numberstyle=\normalsize\color{codegray},
    stringstyle=\color{codepurple},
    basicstyle=\large\tt,
    breakatwhitespace=false,         
    breaklines=true,                 
    captionpos=b,                    
    keepspaces=true,                 
    numbers=left,                    
    numbersep=5pt,                  
    showspaces=false,                
    showstringspaces=false,
    showtabs=false,                  
    tabsize=4,
    xleftmargin=-2em,
    xrightmargin=-2em
}

\lstset{style=mystyle}

\newcommand{\imgmic}{./Img/i8086.jpg}
\newcommand{\imgbs}{./Img/binarysearch.png}
\newcommand{\capOne}{./Capturas-emu/Captura-1.jpg}
\newcommand{\capTwo}{./Capturas-emu/Captura-2.jpg}
\newcommand{\capThree}{./Capturas-emu/Captura-3.jpg}
\newcommand{\capFour}{./Capturas-emu/Captura-4.jpg}
\newcommand{\capFive}{./Capturas-emu/Captura-5.jpg}
\newcommand{\capSix}{./Capturas-emu/Captura-6.jpg}
\newcommand{\capSeven}{./Capturas-emu/Captura-7.jpg}
\newcommand{\capEight}{./Capturas-emu/Captura-8.jpg}

\linespread{1.6}
\begin{document}  
  \begin{titlepage}
    \vfill
    \centering
    {\Huge \textbf{Programación en ensamblador i8086}} \\[4cm]
    \includegraphics[width=\textwidth,height=7cm,keepaspectratio]{\imgmic} \\[4cm]
    \begin{flushleft} 
      {\LARGE \textsc{Tema: Algoritmo de búsqueda binaria}} \\[2ex]
      {\LARGE \textsc{Asignatura: Arquitectura de computadoras}} \\[2ex]
      {\LARGE \textsc{Nombre: Lautaro Galante}} \\[2ex]
      {\LARGE \textsc{Profesor: Víctor Teppaz}} \\[2ex]
      {\LARGE \textsc{Fecha: \Large \today}}
    \end{flushleft}
  \end{titlepage}
  \thispagestyle{plain}
    \begin{center}
      {\Large Como proyecto final de la materia de Arquitectura de computadoras, se nos pidió 
        crear un programa en ensamblador 8086, yo decidí implementar el algoritmo de búsqueda binaria,
        ya que este me haria pensar un poco sobre como hacerlo y asi poder aprender sobre las 
        distintas instrucciones de la arquitectura.
      }
    \end{center}

    \begin{center}
      {\Large 

        En ciencias de la computación y matemáticas, el algoritmo de 
        búsqueda binaria o binary search en inglés, es un algoritmo de 
        búsqueda que encuentra la posición de un valor en un array 
        ordenado. Compara el valor con el elemento en el medio del 
        array, si no son iguales, la mitad en la cual el valor no 
        puede estar es eliminada y la búsqueda continúa en la mitad 
        restante hasta que el valor se encuentre.
      } 
    \end{center}
    
    \begin{center} 
      {\Large 
        La búsqueda binaria es computada en el peor de los casos en 
        un tiempo logarítmico, realizando \(O(\log n) \) comparaciones, 
        donde \textbf{n} es el número de elementos del arreglo y 
        \textbf{log} es el logaritmo.
        La búsqueda binaria requiere solamente \(O(1)\) en espacio, 
        es decir, que el espacio requerido por el algoritmo 
        es el mismo para cualquier cantidad de elementos en el array. 
      }
    \end{center}
    
    \vspace*{1cm}

    {\Large Para que se entienda mejor dejo expresado el algoritmo
     en pseudocódigo:
     \begin{singlespace}
     \begin{algorithm}[H]
      \DontPrintSemicolon
      \SetAlgoLined
      \SetKwFunction{FMain}{busqueda\_binaria}
      \SetKwProg{Fn}{Function}{:}{}
      \Fn{\FMain{$Array$, $n$, $T$}}{
          $Izquierda := 0$\;
          $Derecha := n - 1$\;
          \While{$Izquierda \leq Derecha$}{
              $medio := (Izquierda + Derecha) / 2$\;
              \eIf{$Array[medio] < T$}{
                  $Izquierda := medio + 1$\;
              }{
                  \eIf{$Array[medio] > T$}{
                      $Derecha := medio - 1$\;
                  }{
                      \Return{$medio$}\;
                  }
              }
          }
          \Return{no se encontro}\;
      }
      \end{algorithm} 
      \end{singlespace}
    }
    \newpage
    \vspace*{-2cm}
    \begin{center}
      {\Large Aqui hay una representación gráfica de los pasos que realiza el algoritmo}
    \end{center} 
    \vspace*{1cm}
    \begin{figure}[H]
      \centering
      \hspace{0.5cm}\includegraphics[width=0.8\linewidth,height=0.9\textheight]{\imgbs}
    \end{figure}

    \newpage

    \vspace*{-2cm}
    \begin{center}
      {\Large A continuación explicare que hace cada parte del código en ensamblador}
      \vspace*{2cm}
      
      {\normalsize Bloque 1}
      \begin{lstlisting}[language=8086]
      ORG 100h

      .DATA
          numeros db 1, 2, 3, 4, 5, 6, 7, 8, 9 
          numero db ?
          mensaje db "Ingrese el numero a buscar: ", "$"
          resultado db "El indice del valor ingresado es: ", "$"
          left db 0
          right db 8
          middle db ?
          value db ?
      \end{lstlisting}
    \end{center} 
    \vspace*{1cm}
      {\Large Al inicio del código en la linea \textbf{1} declaro \codeblue{\textbf{ORG}} \textbf{100h} que es una directiva del compilador,
        luego en la linea \textbf{3} defino un sector de datos \codeblue{\textbf{.DATA}} en el cuál tengo un array y distintas variables, 
        el primero es un array \textbf{numeros} que es de tipo \codeblue{\textbf{db}} que tiene una capacidad 
        de \textbf{8 bits} es decir \textbf{1 byte}, este contiene números del \textbf{1} al \textbf{9},
        en la linea \textbf{5} una variable \textbf{numero} del mismo tipo la cuál la inicialize con el simbolo \textbf{?} 
        que significa que aún no se le a asignado un valor, en la linea \textbf{6} una variable string \textbf{mensaje} y en la \textbf{7} 
        \textbf{resultado} para mostrar los mensajes para que se ingrese el número, y para indicar el valor de salida. Luego hay cuatro 
        variables, \textbf{left, rigth, middle y value}, \textbf{left} guarda el primer indice del array que es \textbf{0} 
        y \textbf{right} guarda el inidice \textbf{8} que es el ultimo elemento del array en este caso es el \textbf{9}, en la variable 
        \textbf{middle} y \textbf{value} aún no le asigne ningún valor, luego mas adelante en el codigo se explicara para que son cada variable.
      }   

    \newpage

    \vspace*{-1cm}
    \begin{center}
      {\normalsize Bloque 2}
      \begin{lstlisting}[language=8086]
      .CODE
        MAIN:
          mov ax, numeros
          mov dx, offset mensaje
         
          mov ah, 9h
          int 21h
    
          mov ah, 01h
          int 21h
          sub al, '0'
          mov numero, al
    
          call BINARYSEARCH
          call PRINT
      \end{lstlisting}
    \end{center} 
    \vspace*{0.5cm}
      {\Large 
        En el bloque \textbf{2} de codigo dentro del sector \codeblue{\textbf{.CODE}} se encuentra el procedimiento principal \codeblue{\textbf{.MAIN}}
        que tiene las primeras instrucciones y las llamadas a otros procedimientos, en la linea \textbf{3} se copian 
        los valores del array \textbf{numeros} a el registro entero de 16 bits \codebordo{\textbf{ax}} usando la instrucción 
        \codeblue{\textbf{mov}}, en la linea \textbf{4} se copia la dirección de memoria de la variable \textbf{mensaje}
        utilizando el operador \codeblue{\textbf{offset}} en el registro \codebordo{\textbf{dx}}, en la linea \textbf{6} se copia el valor 
        hexadecimal \textbf{9h} al registro \codebordo{\textbf{ah}} que es la parte alta de \codebordo{\textbf{ax}} esta es una función para 
        imprimir una cadena por consola, en la linea \textbf{7} se ejecuta la interrupción \codeblue{\textbf{int}} \textbf{21h} el \textbf{bios}
        busca el código de función que coincide con la función cargada en el registro \codebordo{\textbf{ah}} y procede a imprimir la cadena 
        \codepurple{\textbf{Ingrese el numero a buscar:}} que se encuentra guardada en la variable \textbf{mensaje}. 
      }
      \par
      {\Large En la linea \textbf{9} se copia el valor de la función \textbf{01h} en el registro \codebordo{\textbf{ah}} esta función es para 
        leer un carácter desde el teclado, en la linea \textbf{10} se ejecuta la interrupción, en la linea \textbf{11} se convierte el valor 
        numérico \textbf{ascci} a decimal y se almacena en el registro \codebordo{\textbf{al}}, 
        luego en la linea \textbf{12} se copia el valor del registro \codebordo{\textbf{al}} en la variable \textbf{numero}, 
        en la linea \textbf{14} se utiliza el operador \codeblue{\textbf{call}} para llamar al procedimiento \codeblue{\textbf{BINARYSEARCH}} 
        que es donde ocurre toda la lógica del algoritmo, y en la linea \textbf{15} se llama al procedimiento \codeblue{\textbf{PRINT}} 
        que lo utilizo para imprimir los valores.
      }
    \vspace*{1cm}
    \begin{center}
      {\normalsize Bloque 3}
      \begin{lstlisting}[language=8086]
        BINARYSEARCH PROC
            mov ax, 0
            mov bl, left
            mov bh, right
            
            SEARCHLOOP:
                cmp bl, bh
                jg SEARCHEND
      \end{lstlisting}
    \end{center} 
      \vspace*{0.5cm}
      {\Large En este bloque \textbf{3} de código definimos el procedimiento \codeblue{\textbf{BINARYSEARCH}}, donde dentro en la linea 
         \textbf{2} limpiamos el registro de \codebordo{\textbf{ax}} copiando el valor \textbf{0}, en la linea \textbf{3} copiamos
         el valor almacenado en la variable \textbf{left} a \codebordo{\textbf{bl}} la parte baja de \codebordo{\textbf{bx}}, en la linea \textbf{4}
         copiamos el valor del la variable \textbf{right} a \codebordo{\textbf{bh}} su parte alta.
      }

      {\Large Luego en la linea \textbf{6} definimos un ciclo \codeblue{\textbf{SEARCHLOOP}}, en el utilizamos la instrucción \codeblue{\textbf{cmp}}
          su función es equivalente a la de una resta, ya que resta el segundo operando al el primer operando, pero no guarda el resultado de la resta,
          sino que cambia las banderas de estado en el registro de banderas, \textbf{CF:} se establece en 1 si el primer operando es menor al segundo, 
          En este caso como el segundo operando es mayor al primero, la bandera \textbf{CF} se setea en \textbf{1}.
      }

      {\Large En la linea \textbf{8} la instrucción \codeblue{\textbf{jg}} corre la ejecución del programa a donde se encuentre la definición 
          de la etiqueta asignada a su derecha, que en este caso es \codeblue{\textbf{SEARCHEND}}, esto sucede si el primer operando es mayor
          al segundo, lo que llevaria a terminar el ciclo ya que no tiene mucho sentido seguir con la búsqueda por que cuando se llega a esta 
          instancia significa que se encontro el indice del valor ingresado.
      }
    \newpage  
    \vspace*{-2cm}
    \begin{center}
      {\normalsize Bloque 4}
      \begin{lstlisting}[language=8086]
            mov al, bh
            add al, bl
            mov cl, 2
            div cl
            
            mov cx, 0
            mov cl, numero
      \end{lstlisting}
    \end{center} 
      \vspace*{0.5cm}
      {\Large Este bloque \textbf{4} de código es una continuación de las instrucciónes que se encuentran dentro del ciclo
        \codeblue{\textbf{SEARCHLOOP}}, en la linea \textbf{1} copiamos el valor de \codebordo{\textbf{bh}} 
        en \codebordo{\textbf{al}}, en la linea \textbf{2} con la instrucción \codeblue{\textbf{add}} sumamos
        el valor de \codebordo{\textbf{bl}} a \codebordo{\textbf{al}}. 
        En la linea \textbf{3} copiamos el valor \textbf{2} al registro bajo \codebordo{\textbf{cl}},
        y en la linea \textbf{4} usamos la intrucción \codeblue{\textbf{div}} para dividir en dos el valor que se encuentra
        en \codebordo{\textbf{al}}, por lo que ahora tendremos dos valores en el registro \codebordo{\textbf{ax}},
        en \codebordo{\textbf{ah}} se ecuentra el residuo y en \codebordo{\textbf{al}} se encuentra el cociente.
      }

      {\Large En la linea \textbf{6} limpiamos el valor del registro \codebordo{\textbf{cx}},
        y en la linea \textbf{7} copiamos el valor del la variable \textbf{numero}, que es el valor que ingresamos 
        teclado.
      }

    \vspace*{1cm}

    \begin{center}
      {\normalsize Bloque 5}
      \begin{lstlisting}[language=8086]
            mov ah, 0
            mov si, ax
            
            mov middle, al
            mov al, numeros[si]
            mov value, al
            
            cmp value, cl
            jne second
            jmp SEARCHEND
      \end{lstlisting}
    \end{center} 

    \vspace*{-1cm}
      \newpage
      {\Large En el bloque \textbf{5} de código, en la linea número \textbf{1} se sobreescribe con cero el valor del registro alto \codebordo{\textbf{ah}}
        para que en el registro entero \codebordo{\textbf{ax}} solo quede en su parte baja \codebordo{\textbf{al}} el resultado de haber 
        dividido la sumatoria de los valores \textbf{left} y \textbf{right} entre \textbf{2}, entonces en la linea \textbf{2} copiamos el valor
        de \codebordo{\textbf{ax}} en \codebordo{\textbf{si}} (\textbf{source index}) que es el índice de la mitad del array \textbf{numeros}.
      }

      {\Large Igualmente necesitamos guardar el valor del índice medio en la variable \textbf{middle}, eso ocurre en la linea \textbf{4}
        donde copiamos el valor de el registro \codebordo{\textbf{al}} ya que necesitamos reescribir ese registro para que en la linea \textbf{5}, 
        utilicemos el \codebordo{\textbf{si}} para iterar dentro del array \textbf{numeros} de la siguiente manera \textbf{numeros[\codebordo{\textbf{si}}]}, 
        entonces con el índice medio dentro del array accedemos al número que se encuentra en esa posición media y lo copiamos en el registro 
        \codebordo{\textbf{al}}, luego en la linea \textbf{6} copiamos ese número en la variable \textbf{value}.
      }
      
      {\Large En la linea \textbf{8} comparamos ese número medio con el número que se ingreso por teclado que ya lo tenemos almacenado en el registro 
      \codebordo{\textbf{cl}} que explique en el bloque \textbf{4}, por lo que si los números no son iguales la instrucción \codeblue{\textbf{jne}}
        salta al procedimiento \codeblue{\textbf{second}}, sino la instrucción \codeblue{\textbf{jmp SEARCHEND}} termina el bucle y se retorna ese 
        número ya que son iguales.
      }
      
      \vspace*{1cm}

    \begin{center}
      {\normalsize Bloque 6}
      \begin{lstlisting}[language=8086]
            second:
                cmp value, cl
                jg third
                
                mov dx, 0
                mov dl, middle
                inc dl
                mov bl, dl
                
                jmp SEARCHLOOP
      \end{lstlisting}
    \end{center} 
  \newpage
  \vspace*{-2cm}
  
  {\Large En el bloque \textbf{6} en la linea \textbf{1} se encuentra la definición del procedimiento \codeblue{\textbf{second}},
    dentro del mismo en la linea \textbf{2} comparamos el valor medio del array, es decir el número que se encuentra en la mitad
    con el número ingresado por teclado, en la linea \textbf{3} si el primer operando es mayor al segundo se salta al procedimiento 
    \codeblue{\textbf{third}}, sino se procede con las operaciones que comienzan en la linea \textbf{5}, se copia el valor \textbf{0} en el 
    registro \codebordo{\textbf{dx}} para limpiarlo, luego en la linea \textbf{6} se copia el indice del valor medio que esta 
    guardado en la variable \textbf{middle} en \codebordo{\textbf{dl}}, luego en la linea \textbf{7} se usa la instrucción \codeblue{\textbf{inc}}
    para incrementar en uno el registro \codebordo{\textbf{dl}}.
  }
  
  {\Large En la linea \textbf{8} se copia el valor incrementado de \codebordo{\textbf{dl}} en el registro \codebordo{\textbf{bl}}
    ya que este corresponde a la parte izquierda del array \textbf{numeros}, para recortar la parte de array en donde no se encuentra 
    valor que se ingreso por teclado.
    Luego en la linea \textbf{10} se salta al principio del loop utilizando la instrucción \codeblue{\textbf{jmp SEARCHLOOP}} para que 
    este se vuelva a iterar.
  }
  \vspace*{1cm}
  \begin{center}
    {\normalsize Bloque 7}
    \begin{lstlisting}[language=8086]      
              third:
                  mov dx, 0
                  mov dl, middle
                  dec dl
                  mov bh, dl
                  
                  jmp SEARCHLOOP
                      
            SEARCHEND:
            RET

        BINARYSEARCH ENDP

    END MAIN
    \end{lstlisting}
  \end{center} 
  \newpage

  {\Large En el bloque \textbf{7} de código en la linea \textbf{1} se define el procedimiento \codeblue{\textbf{third}},
    donde en la linea \textbf{2} limpiamos el registro \codebordo{\textbf{dx}}, en la linea \textbf{3}
    copiamos el valor de la variable \textbf{middle} en \codebordo{\textbf{dl}}, y luego en la linea \textbf{4} 
    utilizamos la instrucción \codeblue{\textbf{dec}} para decrementar en \textbf{1} el valor que se encuentra en 
    \codebordo{\textbf{dl}}, ya que en este caso no necesitamos hacer ninguna comparación, por que el programa llego 
    hasta este procedimiento por que la comparación del procedimiento anterior \codeblue{\textbf{second}} dio que el primer
    operando era mayor al segundo, entonces en este procedimiento solo se realizan las operaciones necesarias, para que en 
    la linea \textbf{5} se copie el nuevo valor de \codebordo{\textbf{dl}} en \codebordo{\textbf{bh}} que es la parte 
    derecha del array \textbf{numeros}, luego en la linea \textbf{7} con la instrucción \codeblue{\textbf{jmp SEARCHLOOP}}
    volvemos a iterar hasta que se encuentra el valor que buscamos, cuando la ejecución llega nuevamente a la comparación 
    que ocurre en el bloque de código número \textbf{3} y el valor del registro \codebordo{\textbf{bl}} es mayor que 
    \codebordo{\textbf{bh}} el bucle \codeblue{\textbf{SEARCHLOOP}} termina y se retorna en la linea \textbf{10} con la 
    instrucción \codeblue{\textbf{RET}}
  }
 \vspace*{2cm}
\begin{center}
    {\normalsize Bloque 8}
    \begin{lstlisting}[language=8086] 
        PRINT:
            mov dl, 13
            mov ah, 2
            int 21h

            mov dl, 10
            mov ah, 2
            int 21h

            mov dx, offset resultado
            mov ah, 9h
            int 21h

            mov dl, middle
            add dl, '0'
            mov ah, 02h
            int 21h         
            mov ax, 4C00h
            int 21h
    \end{lstlisting}
\end{center}
\vspace*{1cm}
{\Large Al retornar, el procedimiento que sigue en el orden de llamadas es el procedimiento \codeblue{\textbf{PRINT}} que sirve para imprimir 
  el índice del número que ingresamos por teclado, que es la posición donde se encuentra ese número en el array.
  En la linea \textbf{2} del bloque \textbf{8} de código se copia el valor \textbf{13} en el registro \codebordo{\textbf{dl}}, en la linea \textbf{3} se 
  copia el valor \textbf{2} en \codebordo{\textbf{ah}}, en la linea \textbf{4} se usa la instrucción \codeblue{\textbf{int}} \textbf{21h}, que en conjunto
  estas tres instrucciónes lo que hacen es aplicar un retorno de carro, para que en la linea \textbf{6, 7 y 8} se aplique un salto de linea en la consola, 
  para que el resultado no se imprima pegado a la impresión anterior. 
 }

 {
   \Large En la linea \textbf{10} se copia la dirección de memoria de la variable \textbf{resultado} usando la instrucción \codeblue{\textbf{offset}} en el registro
  \codebordo{\textbf{dx}}, en la linea \textbf{11} se copia el valor de la función \textbf{9h} en \codebordo{\textbf{ah}}, luego se usa la interrupción 
  \codeblue{\textbf{int}} \textbf{21h} para imprimir la cadena que se guarda en la variable resultado que es \codepurple{\textbf{El indice del valor ingresado es: }},
  en la linea \textbf{14} se copia el indice del número buscado que esta en \textbf{middle} al registro \codebordo{\textbf{dl}}, en la linea \textbf{15} se convierte 
  ese valor \textbf{ascci} a decimal, para que en la linea \textbf{16 y 17} se imprima el indice del número buscado por consola, en la linea \textbf{18 y 19} utilizo 
  una instrucción para terminar la ejecución del programa.
 }

 {\Large 
     A continuación mostrare unas capturas de pantalla de la ejecución del programa en el emulador emu8086.
 }
    \newpage
    \begin{center}
      \includegraphics[width=\textwidth]{\capOne}
    \end{center}
    \vspace*{1cm}
    \begin{center}
      \includegraphics[width=\textwidth]{\capTwo}
    \end{center}
    \newpage

    \begin{center}
      \includegraphics[width=\textwidth]{\capThree}
    \end{center}
    \vspace*{1cm}
    \begin{center}
      \includegraphics[width=\textwidth]{\capFour}
    \end{center}
    \newpage

    \vspace*{-2cm}
    \begin{center}
      \includegraphics[width=\textwidth]{\capFive}
    \end{center}
    \begin{center}
      \includegraphics[width=\textwidth]{\capSix}
    \end{center}
    \newpage

    \begin{center}
      \includegraphics[width=\textwidth]{\capSeven}
    \end{center}
    \begin{center}
      \includegraphics[width=\textwidth]{\capEight}
    \end{center}
\end{document}
